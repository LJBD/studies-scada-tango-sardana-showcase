\clearpage
\section{Opis realizacji}
\label{sec:opis_realizacji}

\quad Realizacja zadań projektowych opisanych w rozdziale \ref{sec:koncepcja} została przeprowadzona przy użyciu wszystkich technologii, które są wymienione w rozdziałach \ref{sec:spec_sprzet} i \ref{sec:spec_prog}. Można ją podzielić na trzy główne części, które zostaną przedstawione w kolejnych podrozdziałach.

\subsection{Konfiguracja wirtualnego środowiska}
\label{sub:konfiguracja}
\quad Pierwszym elementem, który należało opanować było uruchomienie wirtualnej maszyny ,,TangoBox9'', a następnie konfiguracja wszystkich elementów niezbędnych do realizacji postawionego celu. Kolejne kroki postępowania zostały przedstawione poniżej:

\begin{enumerate}
	\item Sprawdzenie funkcjonowania systemu Tango: głównej bazy danych, poszczególnych ,,serwerów urządzeń'' (ang. device servers) oraz samych urządzeń. W tym celu użyto dwóch aplikacji stanowiących pakiet do zarządzania systemem Tango: Astor i Jive. Weryfikacja poprawności polegała na uruchomieniu obu aplikacji i wizualnym sprawdzeniu stanów interesujących z punktu widzenia projektu komponentów systemu.
	\item Sprawdzenie funkcjonowania dodatków do systemu Tango: archiwizacji, bibliotek Taurus i Sardana. Należało uruchomić odpowiednie aplikacje (,,jhdbconfigurator'' w przypadku archiwizacji, ,,taurusgui'' - Taurusa i ,,Sardemo'' - Sardany) i sprawdzić, czy nie wyrzucają jakiś błędów.
	\item Uruchomienie serwerów urządzeń zarządzających silnikami. Klasa ,,Motors'' jest częścią oprogramowania ,,Pool'', które należy do pakietu Sardana. Program Astor umożliwia włączenie serwera.
	\item Konfiguracja silników. Zostały ustawione następujące elementy:
	\begin{enumerate}
		\item urządzenie ,,motor/motctrl01/1'':
		\begin{enumerate}
			\item zakres wartości położenia: -120, 120,
			\item progi alarmowe: -110, 110,
			\item progi ostrzeżeń: -100, 100,
			\item pozycje czujników krańcowych - takie, jak zakresy wartości położenia,
			\item prędkość: 10,
			\item przyspieszenie: 0,5.
		\end{enumerate}
		\item urządzenie ,,motor/motctrl01/2'':
		\begin{enumerate}
			\item zakres wartości położenia: -120, 120,
			\item progi alarmowe: -110, 110,
			\item progi ostrzeżeń: -100, 100,
			\item pozycje czujników krańcowych - takie, jak zakresy wartości położenia,
			\item prędkość: 100,
			\item przyspieszenie: 40.
		\end{enumerate}
		\item urządzenie ,,motor/motctrl01/3'':
		\begin{enumerate}
			\item zakres wartości położenia: -50, 50,
			\item progi alarmowe: -45, 45,
			\item progi ostrzeżeń: -40, 40,
			\item pozycje czujników krańcowych - takie, jak zakresy wartości położenia,
			\item prędkość: 10,
			\item przyspieszenie: 0,1.
		\end{enumerate}
		\item urządzenie ,,motor/motctrl01/4'':
		\begin{enumerate}
			\item zakres wartości położenia: -90, 140,
			\item progi alarmowe: -85, 130,
			\item progi ostrzeżeń: -80, 125,
			\item pozycje czujników krańcowych - takie, jak zakresy wartości położenia,
			\item prędkość: 100,
			\item przyspieszenie: 1.
		\end{enumerate}
	\end{enumerate}
\end{enumerate}

\subsection{Opracowanie aplikacji eksperckiej i operatorskiej}


\subsection{Testy}