\clearpage
\section{Podstawa teoretyczna}
\label{sec:podstawa}


\hspace{2em}Systemy SCADA (ang. Supervisory control and data acquisition) to system do zdalnego monitorowania i sterowania procesem techologicznym. Głównymi funkcjami systemów SCADA są:

\begin{itemize}
	\item zbieranie aktualnych danych (pomiarów)
	\item wizualizacja danych
	\item sterowanie procesem
	\item alarmowanie
	\item archiwizacja danych	
\end{itemize}

\hspace{2em}Systemy te dają możliwość współpracy ze sterownikami PLC, regulatorami mikroprocesorowymi i innymi urządzeniami. Pozwalają na realizację zdecentralizowanych systemów automatyki przemysłowej. Systemy SCADA pozwalają na uzyskanie szybkiego wglądu w faktyczny stan urządzeń produkcyjnych i wykonawczych. Są one doskonałym sposobem nie tylko na zamianę języka maszyn na język ludzi, ale także umożliwiają szybką lokalizację alarmów, podstawowe logowanie danych czy też automatyczną reakcję na określone sygnały pochodzące z urządzeń. System SCADA w warstwie graficznej odpowiada za jednoznaczne zaprezentowanie dynamicznie zmieniającej się informacji. Jednocześnie zdefiniowane przez użytkownika algorytmy logiczne przyspieszają i wspomagają operatora w jego pracy. System SCADA jest także podstawowym źródłem danych dla systemów nadrzędnych i przemysłowych baz danych.

