%%%%%%%%%%%%%%%%%%%%%%%%%%%%%%%%%%%%%%%%%
% Journal Article
% LaTeX Template
% Version 1.3 (9/9/13)
%
% This template has been downloaded from:
% http://www.LaTeXTemplates.com
%
% Original author:
% Frits Wenneker (http://www.howtotex.com)
%
% License:
% CC BY-NC-SA 3.0 (http://creativecommons.org/licenses/by-nc-sa/3.0/)
%
%%%%%%%%%%%%%%%%%%%%%%%%%%%%%%%%%%%%%%%%%

%----------------------------------------------------------------------------------------
%	PACKAGES AND OTHER DOCUMENT CONFIGURATIONS
%----------------------------------------------------------------------------------------

\documentclass[twoside]{article}

\usepackage{amssymb}
\usepackage[polish]{babel}
\usepackage{polski}

\usepackage{lipsum} % Package to generate dummy text throughout this template

\usepackage[sc]{mathpazo} % Use the Palatino font
\usepackage[T1]{fontenc} % Use 8-bit encoding that has 256 glyphs
\usepackage[utf8]{inputenc}
\selectlanguage{polish}

\linespread{1.05} % Line spacing - Palatino needs more space between lines
\usepackage{microtype} % Slightly tweak font spacing for aesthetics

\usepackage[hmarginratio=1:1,top=32mm,columnsep=20pt]{geometry} % Document margins
%\usepackage{multicol} % Used for the two-column layout of the document
\usepackage[hang, small,labelfont=bf,up,textfont=it,up]{caption} % Custom captions under/above floats in tables or figures
\usepackage{booktabs} % Horizontal rules in tables
%\usepackage{float} % Required for tables and figures in the multi-column environment - they need to be placed in specific locations with the [H] (e.g. \begin{table}[H])
\usepackage{hyperref} % For hyperlinks in the PDF

%\usepackage{lettrine} % The lettrine is the first enlarged letter at the beginning of the text
%\usepackage{paralist} % Used for the compactitem environment which makes bullet points with less space between them

\usepackage{abstract} % Allows abstract customization
\renewcommand{\abstractnamefont}{\normalfont\bfseries} % Set the "Abstract" text to bold
\renewcommand{\abstracttextfont}{\normalfont\small\itshape} % Set the abstract itself to small italic text

\usepackage{titlesec} % Allows customization of titles
\renewcommand\thesection{\Roman{section}} % Roman numerals for the sections
\renewcommand\thesubsection{\Roman{subsection}} % Roman numerals for subsections
\titleformat{\section}[block]{\large\scshape\centering}{\thesection.}{1em}{} % Change the look of the section titles
\titleformat{\subsection}[block]{\large}{\thesubsection.}{1em}{} % Change the look of the section titles

\usepackage{graphicx} % Required for including images
\graphicspath{{Grafika/}} % Set the default folder for images

\usepackage{fancyhdr} % Headers and footers
\pagestyle{fancy} % All pages have headers and footers
\fancyhead{} % Blank out the default header
\fancyfoot{} % Blank out the default footer
\fancyhead[C]{AGH $\bullet$ WSS i SCADA $\bullet$ Ciszewski, Dudek, Mucha $\bullet$ 2016 r.} % Custom header text
\fancyfoot[RO,LE]{\thepage} % Custom footer text

\usepackage[nottoc,numbib]{tocbibind}
\usepackage{adjustbox} %MINE: handles rotated images (like the screenshots of GUIs)
\usepackage{titletoc} %MINE: adding spacing in table of contents


%----------------------------------------------------------------------------------------
%	TITLE SECTION
%----------------------------------------------------------------------------------------

\title{\vspace{-15mm}\fontsize{24pt}{10pt}\selectfont\textbf{Wizualizacja danych eksperymentalnych w symulowanym środowisku linii pomiarowej w synchrotronie}} % Article title

\author{
\large
\textsc{Michał Ciszewski, Łukasz Dudek, Krystian Mucha}\\[3mm] % Your name
\textsc{Na przedmiot: Wielopoziomowe struktury sterowania i systemy SCADA}\\[3mm] % Your email address
\normalsize Prowadzący: prof. dr hab inż. Witold Bryski, mgr. inż. Andrzej Latocha\\[9mm]
\textsc{Akademia Górniczo - Hutnicza w Krakowie} % Your institution
}
\date{10 stycznia 2016 r.}

%----------------------------------------------------------------------------------------

\begin{document}
	
\begin{figure}
	\centering
	\includegraphics[width=0.9\linewidth]{Grafika/agh_logo}
	\label{fig:agh-logo}
\end{figure}

\maketitle % Insert title

\thispagestyle{fancy} % All pages have headers and footers

\clearpage


%----------------------------------------------------------------------------------------
%	ABSTRACT
%----------------------------------------------------------------------------------------
\vspace{10mm}
\begin{abstract}

\noindent Niniejszy raport podsumowuje prace nad projektem na przedmiot ,,Wielopoziomowe struktury sterowania i systemy SCADA'' na Akademii Górniczo - Hutniczej im. S. Staszica w Krakowie. Dotyczy przygotowania dwóch sposobów wizualizacji sterowania zespołami silników w symulowanym środowisku linii badawczej w synchrotronie - aplikacji eksperckiej i klienckiej. Zostały omówione podstawy teoretyczne, środowisko programowo - sprzętowe i jego konfiguracja oraz sama realizacja projektu. W ostatnim rozdziale znajduje się instrukcja użytkowania przygotowanych aplikacji.

\end{abstract}

\smallskip
\noindent \textbf{Słowa kluczowe:} SCADA, zespoły silników, Tango Controls, Sardana, Taurus.

\tableofcontents

%----------------------------------------------------------------------------------------
%	ARTICLE CONTENTS
%----------------------------------------------------------------------------------------

\clearpage
\section{Podstawa teoretyczna}
\label{sec:podstawa}


\clearpage
\section{Koncepcja realizacji}
\label{sec:koncepcja}

\quad Postawiony w temacie projektu problem jest dwojakiej natury. Po pierwsze, należy zdefiniować system sterowania linii badawczej w synchrotronie, określić w nim rolę silników i zaproponować programowe sposoby wizualizacji ich pracy. Po drugie, jak było wspomniane w początkowym opisie projektu, trzeba zastanowić się nad możliwościami synchronizacji pracy poszczególnych osi takich silników krokowych, które, jak było wspomniane w rozdziale \ref{sec:podstawa}, ruszają wieloma elementami na liniach badawczych.

Należy również uwzględnić fakt, iż w normalnym środowisku pracy przy linii badawczej w czasie jej pracy mogą nią sterować dwie osoby. Jedna z nich to operator - klient synchrotronu, który przychodzi zrealizować pewne badania na danej linii. Drugą osobą jest ekspert ze strony synchrotronu - nie ma on obowiązku posiadania pełnej znajomości fizyki stojącej za funkcjonowaniem linii, ale powinien dobrze znać jej system sterowania, ponieważ może on mieć wpływ na serce synchrotronu, czyli pierścień akumulacyjny.

Dodatkowo, co jest dość oczywiste, przy tworzeniu projektu autorzy nie mieli dostępu do prawdziwej linii badawczej. W związku z tym wykorzystano środowisko symulacyjne, które jest dostarczane razem z narzędziami do zarządzania nim. Umożliwiło to dokładną konfigurację oraz testowanie różnych zestawów parametrów.

Na podstawie wyżej wymienionych zagadnień sformułowane zostały wymagania stawiane przed projektem:

\begin{enumerate}
	\item Należy skonfigurować różne urządzenia symulujące pracę rzeczywistych silników z uwzględnieniem takich parametrów, jak prędkość, przyspieszenie lub położenie czujników krańcowych. Liczba takich symulacyjnych napędów powinna wynosić minimum 3.
	\item Należy znaleźć sposób synchronizacji przynajmniej dwóch z tych silników. Przez synchronizację rozumie się wykonywanie zaplanowanego ruchu w jednostajny sposób i w jednym czasie. Bez względu na środowisko symulacyjne, zaleca się zastosowanie rozwiązania czysto programowe. Jest to związane z faktem, iż używany zestaw narzędzi do zarządzania systemem sterowania jest obiektowy, co pozwala na hierarchiczność urządzeń. W ten sposób można tworzyć urządzenia nadrzędne, które będą kontrolować rzeczywiste napędy.
	\item Należy zaproponować dwa odrębne sposoby wizualizacji danych eksperymentalnych, czyli w tym przypadku pozycji symulowanych silników.
	\item Należy przygotować dwie aplikacje: dla eksperta i zwykłego użytkownika. Pierwsza powinna dawać pełną kontrolę nad wszystkimi parametrami, za wyjątkiem stałych parametrów rzeczywistych silników (takich jak prędkość czy przyspieszenie). Druga powinna umożliwiać tylko podgląd stanów poszczególnych silników oraz wprowadzanie drobnych korekt w ich pozycjach.
\end{enumerate}

Taki zestaw założeń projektowych został przyjęty w czasie jego realizacji.

%\clearpage
\section{Specyfikacja sprzętu}
\label{sec:spec_sprzet}

\quad W przypadku tak stricte programowego projektu, jak opisywany w niniejszym raporcie ciężko jest wyróżnić jakąś szczególną specyfikację potrzebnego sprzętu. Do realizacji wykorzystywaliśmy tylko komputer klasy PC z oprogramowaniem opisanym w rozdziale \ref{sec:spec_prog}.

To, o czym jednak należy tutaj wspomnieć, to używane zwykle w synchrotronach zestawy silników krokowych, które mają specyficzne możliwości konfiguracyjne oraz mogą być prosto zintegrowane z rozproszonym systemem sterowania zarządzanym przy pomocy narzędzi Tango. Najczęściej stosowanym rozwiązaniem jest platforma programowo - sprzętowa IcePAP.

\clearpage
\section{Specyfikacja programowo-sprzętowa}
\label{sec:spec_prog}

\subsection{Sprzęt}
\quad W przypadku tak stricte programowego projektu, jak opisywany w niniejszym raporcie ciężko jest wyróżnić jakąś szczególną specyfikację potrzebnego sprzętu. Do realizacji wykorzystywaliśmy tylko komputer klasy PC z oprogramowaniem opisanym w podrozdziale \ref{sub:plat_prog}.

To, o czym jednak należy tutaj wspomnieć, to używane zwykle w synchrotronach zestawy silników krokowych, które mają specyficzne możliwości konfiguracyjne oraz mogą być prosto zintegrowane z rozproszonym systemem sterowania zarządzanym przy pomocy narzędzi Tango. Najczęściej stosowanym rozwiązaniem jest platforma programowo - sprzętowa IcePAP \cite{web:icepap, proc:icepap}.

\subsection{Platformy programowe}
\label{sub:plat_prog}

\begin{enumerate}
	\item \textbf{Tango}
	
	\hspace{2em}System sterowania TANGO jest darmowym zestawem narzędzi do sterowania dowolnego rodzaju sprzętem lub oprogramowaniem oraz do budowy systemów SCADA (ang. Supervisory Control And Data Acquisition). Jest to oprogramowanie typu open-source wykorzystywane do sterowania synchrotronami, laserami oraz innymi eksperymentami fizycznymi. TANGO jest aktywnie rozwijane przez TANGO Consortium.
	
	\hspace{2em}TANGO jest rozproszonym systemem sterowania. Oznacza to, że może działać zarówno na jednej maszynie, jak i na wielu. Wykorzystuje dwa protokoły sieciowe - COBRA oraz Zeromq. Podstawowym modelem komunikacji jest model klient-serwer. Komunikacja pomiędzy klientem i serwerem może być synchroniczna, asynchroniczna (COBRA) oraz sterowana zdarzeniem (Zeromq) \cite{TangoWiki}. Schemat ideowy całości oprogramowania dostępnego w ramach projektu Tango jest pokazany na rys. \ref{fig:tango-outline}. Jest na nim zaprezentowany podział na pakiety do budowania aplikacji klienckich, programy służące do zarządzania, archiwizację, centralną bazę danych (pod sekcją TANGO HOST) oraz mnogość urządzeń skomunikowanych z resztą systemu przez dedykowane interfejsy programowej magistrali - serwery urządzeń (ang. Device Servers).
	
	\begin{figure}[hpt]
		\centering
		\includegraphics[width=1\linewidth]{Grafika/tango_outline}
		\caption{Schemat ideowy różnych komponentów systemu sterowania Tango. Wykonanie schematu: Piotr Goryl, NCPS Solaris.}
		\label{fig:tango-outline}
	\end{figure}
	
	\hspace{2em}System TANGO jest opakowaniem (ang. wrapper) dla protokołu COBRA zapewniającym przyjazne dla użytkownika API. Dzięki takiemu podejsciu wiele detali związanych z nawiązywaniem i utrzymywaniem połączenia pomiędzy urządzeniami jest niewidoczna dla użytkownia, co pozwala na szybsze i łatwiejsze rozbudowywanie systemu sterowania.
	
	\hspace{2em}TANGO wykorzystuje MySQL jako bazę danych do trzymania informacji. Informacjami takimi mogą byc np. nazwy urządzeń, adresy sieciowe, listy urządzeń itp. MySQL jest relacyjną bazą danych implementującą podzbiór SQLa. System TANGO jest wspierany przez 4 platformy: Linux, Windows NT, Solaris oraz HP-UX.
	
	\begin{enumerate}
		\item \textbf{Taurus}
		
		\hspace{2em}Taurus jest platformą programistyczną dla Pythona służącą do sterowania oraz akwizycji danych w zastosowaniach naukowych oraz przemysłowych. Pozwala na szybkie i proste tworzenie interfejsów użytkownika. Jest to częśc pakietu programistycznego Sardana. Taurus posiada bogatą bibliotekę dzięki czemu stworzone GUI (ang. Graphical User Interface)  może zawierać wiele różnych komponentów takich jak wykresy, tabele, przyciski itp. Celem biblioteki Taurus jest zapewnienie przyjaznego API dla użytkownika oraz przyspieszenie procesu rozwijania aplikacji bazowanych na TANGO.
				
		
		\item \textbf{Sardana}
		
		\hspace{2em}Sardana to pakiet oprogramowania do nadzoru, kontroli i akwizycja danych w zastosowaniach naukowych. Jej celem jest zredukowanie kosztów oraz czasu potrzebnych do projektowania, rozwijania oraz utrzymywania systemów SCADA. Rozwój Sardany zapoczątkowany został przy synchrotronie ALBA, a dzisiaj jest wspierany przez wiekszą społeczność,  w skład której wchodzą liczne laboratoria oraz inne jednostki (ALBA, DESY, MaxIV, Solaris, ESRF).
		Sardana jest bazowana na systemie TANGO i wykorzystuje bibliotekę Taurus umożliwiającą programowanie i konfigurację interfejsu użytkownika \cite{Sardana}.
		
		\item \textbf{Narzędzia pakietu TANGO}
		
		\hspace{2em}Najważniejszymi narzędziami pakietu TANGO są Jive oraz Astor. Pierwsze z nich jest wykorzystywane do zarządzania urządzeniami - ich właściwościami, okresem odpytywania, nazwami, wyświetlanymi parametrami, itp. Zostało zaprezentowane na rys. \ref{fig:jive}.
		
		\begin{figure}[pth]
			\centering
			\includegraphics[width=1\linewidth]{Grafika/jive}
			\caption{Zrzut ekranu pokazujący aplikację Jive.}
			\label{fig:jive}
		\end{figure}
		
		\item \textbf{TangoBox9} (maszyna wirtualna)
		
		W związku z polityką udostępniania oprogramowania na otwartych licencjach, konsorcjum Tango wypuściło gotową wirtualną maszynę, która zawiera wszystkie narzędzia pakietu Tango zainstalowane na systemie Ubuntu 14.04 64-bit. Są tam uwzględnione również wszystkie poboczne projekty, które ułatwiają korzystanie z systemu i dostarczają dodatkowych opcji potrzebnych systemowi sterowania, np. archiwizacji, logowania czy tworzenia interfejsów użytkownika. W tym systemie dostarczono w pełni funkcjonalne symulacyjne środowisko systemu sterowania synchrotronem z uwzględnieniem jednej linii badawczej. Obecne są tam również podstawowe aplikacje będące częścią bazowej paczki Tango, takie jak: Astor (służy do zarządzania serwerami urządzeń), Pogo (generator kodu serwerów), Jive czy AtkMoni (umożliwiają konfigurację i testowanie samych urządzeń).
		
	\end{enumerate}
	\item \textbf{Python}
	
	\hspace{2em}Python to wysokopoziomowy język programowania ogólnego przeznaczenia. Posiada rozbudowane pakiety bibliotek standardowych. Ideą przewodnią Pythona jest czytelność i klarowność kodu źródłowego. Jego składnia cechuje się przejrzystością i zwięzłością. W Pythonie możliwe jest programowanie obiektowe, programowanie strukturalne i programowanie funkcyjne. Typy sprawdzane są dynamicznie, a do zarządzania pamięcią stosuje się garbage collection. Python rozprowadzany jest na otwartej licencji umożliwiając także zastosowanie do zamkniętych komercyjnych projektów \cite{Python}.
	
	\begin{enumerate}
		\item \textbf{PyCharm}
		
		\hspace{2em}PyCharm to zintegrowane środowisko programistyczne (IDE) dla języku programowania Python firmy JetBrains. Zapewnia m.in.: edycję i analizę kodu źródłowego, graficzny debugger, uruchamianie testów jednostkowych, integrację z systemem kontroli wersji. Wspiera także programowanie i tworzenie aplikacji internetowych w Django.
		
		\hspace{2em}Jest oprogramowaniem wieloplatformowym pracującym na platformach systemowych: Microsoft Windows, Linux oraz OS X. Wydawany jest w wersji Professional Edition, które jest oprogramowaniem zamkniętym oraz w wersji darmowej Community Edition, które pozbawione jest jednak niektórych funkcjonalności w porównaniu z wersją komercyjną \cite{PyCharm}.
		
	\end{enumerate}
	\item \textbf{QtDesigner}
	
	\hspace{2em}Jest to narzędzie slużące do projektowania graficznego interfejsu użytkownika (GUI), zawarte w wieloplatformowym środowisku programistycznym Qt Creator. Jego wygląd jest pokazany na rys. \ref{fig:taurusdesigner}. Jak widać, można do niego zaimportować zestaw widżetów z biblioteki Taurus - wtedy komponowanie graficzne aplikacji klienckich w systemie Tango staje się bardzo proste. Podstawowe funkcje tego pakietu to:
	\begin{enumerate}
		\item tworzenie różnych typów GUI (głównych okien, pojedynczych paneli, okienek dialogowych, itp.),
		\item dodawanie różnych rodzajów widżetów,
		\item układanie graficzne wybranych elementów,
		\item edytowanie ich właściwości,
		\item zapis w formacie .ui, umożliwiającym późniejsze importowanie do aplikacji w języku C++ lub Python bądź bezpośrednią konwersję na kod w tym drugim języku.
	\end{enumerate}
\end{enumerate}

\begin{figure}[ht]
	\begin{adjustbox}{addcode={
				\begin{minipage}{\width}}
				{\caption{Zrzut ekranu pokazujący aplikację QtDesigner.}\label{fig:taurusdesigner}
				\end{minipage}},rotate=90,center}
			\includegraphics[scale=.45]{Grafika/taurus_designer}
		\end{adjustbox}
	\end{figure}

\clearpage
\section{Opis realizacji}
\label{sec:opis_realizacji}

\quad Realizacja zadań projektowych opisanych w rozdziale \ref{sec:koncepcja} została przeprowadzona przy użyciu wszystkich technologii, które są wymienione w rozdziałach \ref{sec:spec_sprzet} i \ref{sec:spec_prog}. Można ją podzielić na trzy główne części, które zostaną przedstawione w kolejnych podrozdziałach.

\subsection{Konfiguracja wirtualnego środowiska}
\label{sub:konfiguracja}
\quad Pierwszym elementem, który należało opanować było uruchomienie wirtualnej maszyny ,,TangoBox9'', a następnie dokonanie konfiguracji wszystkich elementów niezbędnych do realizacji postawionego celu.

\subsection{Opracowanie aplikacji eksperckiej i operatorskiej}


\subsection{Testy}

\clearpage
\section{Opis załączonego kodu}
\label{sec:opis_kodu}

\quad Załączone do niniejszego raportu archiwum w formacie ZIP, stanowiące techniczno - implementacyjną część projektu stanowi zrzut repozytorium, które zostało założone na potrzeby projektu. Zawiera następujące elementy:
\begin{enumerate}
	\item Katalog ,,OperatorGUI'':
	\begin{enumerate}
		\item Plik Pythona ,,editable\_mano\_meter.py'' zawierający definicję klasy wykresów kołowych użytych w aplikacji operatora.
		\item Plik Pythona ,,operator\_gui.py'' zawierający definicję całej aplikacji operatora.
		\item Plik ,,operator\_gui.ui'' będący podstawą aplikacji operatora utworzoną przy pomocy programu QtDesigner.
	\end{enumerate}
	\item Katalog ,,Raport'' zawierający niniejszy raport oraz jego pliki źródłowe w języku LaTeX. Ten opis nie zawiera jego dokładnej zawartości, jako że nie jest on częścią kodu napisanego na potrzeby projektu.
	\item Plik Pythona ,,\_\_init\_\_.py'' potrzebny Pythonowi, aby odpowiednio rozpoznał pliki w tym języku znajdujące się wewnątrz tego katalogu.
	\item Pliki konfiguracyjne aplikacji eksperckiej: ,,config.xml'' oraz ,,config.py''.
	\item Skrypt ,,ExpertGUI'' służący do uruchomienia aplikacji eksperckiej.
	\item Plik ,,macro\_sequence\_motors\_showcase.xml'' stanowiący zapis sekwencji makr.
	\item Plik ,,motors\_positions.pck'' stanowiący zapis konfiguracji wykresu w aplikacji eksperckiej.
	\item Plik ,,README.md'' opisujący skrótowo zawartość katalogu projektowego.
	\item Plik ,,wizard.log'' zawierający podsumowanie generacji ustawień aplikacji eksperckiej.
	\item Pliki konfiguracyjne repozytorium.
\end{enumerate}

\clearpage
\section{Instrukcja użytykownika}
\label{sec:manual}

Jak zostało wcześniej wspomniane, możliwość interakcji z systemem silników została przewidziana w formie dwóch aplikacji. Niniejszy rozdział został poświęcony 

%----------------------------------------------------------------------------------------
%	REFERENCE LIST
%----------------------------------------------------------------------------------------

\clearpage
\bibliographystyle{alpha}
\begin{thebibliography}{99} % Bibliography - this is intentionally simple in this template
	
\bibitem{synchrotron_uj_edu}
Strona internetowa NCPS Solaris w Krakowie.
\newblock \texttt{http://www.synchrotron.uj.edu.pl/} .
\newblock Stan na: 10.01.2016 r.

\bibitem{web:exp_line}
Outline of optical diagnostic system.
\newblock \texttt{http://www.spring8.or.jp/en/about\_us/whats\_sp8\\/facilities/accelerators/upgrading/operation\_source/\\acc\_bd\_optical\_diag\_overview.html}
\newblock Zarządzane przez: SPring-8.
\newblock Stan na: 10.01.2016 r.

\bibitem{web:tango}
Strona internetowa Tango Controls.
\newblock \texttt{http://www.tango-controls.org/} .
\newblock Zarządzane przez: Tango Consortium.
\newblock Stan na: 10.01.2016 r.

\bibitem{web:taurus}
Dokumentacja zestawu narzędzi Taurus.
\newblock \texttt{http://www.taurus-scada.org/en/stable/} .
\newblock Zarządzane przez: ALBA Synchrotron.
\newblock Stan na: 10.01.2016 r.

\bibitem{web:sardana}
Dokumentacja zestawu narzędzi Sardana.
\newblock \texttt{http://sardana.readthedocs.org/en/stable/} .
\newblock Zarządzane przez: ALBA Synchrotron.
\newblock Stan na: 10.01.2016 r.

\bibitem{web:icepap}
Strona internetowa poświęcona zestawowi silników IcePAP.
\newblock \texttt{http://www.esrf.eu/\\Instrumentation/DetectorsAndElectronics/icepap} .
\newblock Zarządzane przez: ESRF.
\newblock Stan na: 10.01.2016 r.

\bibitem{proc:icepap}
Janvier, N., Clement, J. M., Fajardo, P.,  
\newblock{\em IcePAP: An Advanced Motor Controller for Scientific Applications in Large User Facilities}.
\newblock W materiałach: ICALEPCS 2013, San Francisco, Kalifornia, USA.

\bibitem{TangoWiki}
System TANGO
\newblock \texttt{https://en.wikipedia.org/wiki/TANGO} .
\newblock Stan na: 13.01.2016 r.

\bibitem{Sardana}
Sardana Home Page
\newblock \texttt{http://www.sardana-controls.org/en/stable/} .
\newblock Stan na: 13.01.2016 r.

\bibitem{Python}
Python
\newblock \texttt{https://pl.wikipedia.org/wiki/Python} .
\newblock Stan na: 13.01.2016 r.

\bibitem{PyCharm}
PyCharm
\newblock \texttt{https://pl.wikipedia.org/wiki/PyCharm} .
\newblock Stan na: 13.01.2016 r.

%\bibitem{GMeansExplanation}
%Użytkownik 'ashenfad'.
%\newblock{\em Divining the 'K' in K-means Clustering}.
%\newblock \texttt{http://blog.bigml.com/2015\\/02/24/divining-the-k-in-k-means-clustering/}. 
%\newblock Dodane: 24.02.2015 r.
%\newblock Stan na: 05.12. 2015 r.


\end{thebibliography}

\end{document}
