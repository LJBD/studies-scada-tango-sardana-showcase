\clearpage
\section{Opis załączonego kodu}
\label{sec:opis_kodu}

\quad Załączone do niniejszego raportu archiwum w formacie ZIP, stanowiące techniczno - implementacyjną część projektu stanowi zrzut repozytorium, które zostało założone na potrzeby projektu. Zawiera następujące elementy:
\begin{enumerate}
	\item Katalog ,,OperatorGUI'':
	\begin{enumerate}
		\item Plik Pythona ,,editable\_mano\_meter.py'' zawierający definicję klasy wykresów kołowych użytych w aplikacji operatora.
		\item Plik Pythona ,,operator\_gui.py'' zawierający definicję całej aplikacji operatora.
		\item Plik ,,operator\_gui.ui'' będący podstawą aplikacji operatora utworzoną przy pomocy programu QtDesigner.
	\end{enumerate}
	\item Katalog ,,Raport'' zawierający niniejszy raport oraz jego pliki źródłowe w języku LaTeX. Ten opis nie zawiera jego dokładnej zawartości, jako że nie jest on częścią kodu napisanego na potrzeby projektu.
	\item Plik Pythona ,,\_\_init\_\_.py'' potrzebny Pythonowi, aby odpowiednio rozpoznał pliki w tym języku znajdujące się wewnątrz tego katalogu.
	\item Pliki konfiguracyjne aplikacji eksperckiej: ,,config.xml'' oraz ,,config.py''.
	\item Skrypt ,,ExpertGUI'' służący do uruchomienia aplikacji eksperckiej.
	\item Plik ,,macro\_sequence\_motors\_showcase.xml'' stanowiący zapis sekwencji makr.
	\item Plik ,,motors\_positions.pck'' stanowiący zapis konfiguracji wykresu w aplikacji eksperckiej.
	\item Plik ,,README.md'' opisujący skrótowo zawartość katalogu projektowego.
	\item Plik ,,wizard.log'' zawierający podsumowanie generacji ustawień aplikacji eksperckiej.
	\item Pliki konfiguracyjne repozytorium.
\end{enumerate}